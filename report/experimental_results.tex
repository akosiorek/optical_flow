\newpage
\section{Experimental Results}
In the following, the implementation of the algorithm has been evaluated with regard to its efficiency, as well as to its accurateness with regard to ground truth.

\subsection{Quantitative:}
One of the primary goals of this project was to evaluate the real-time capabilities of this algorithm.
This section concerns itself exclusively with the evaluation of execution speed of the algorithm with respect to real-time.

As a simplification, real-time shall be defined as following;
\begin{defwrp}
	Given a stream of events, with duration $t_d$, we say it is processed in real-time, if the total processing time $t_p$ is smaller or equal to $t_d$.
\end{defwrp}

This definition is simplified assumption as real-time constraint as it is oblivious to the fact that many events could occur in a very short time period.
However, since the implementation, both the sequential and parallel one, are quantizing events to discrete slices of events, the effects can be treated as negligible.
The evaluation speed hereby primarily depends on the temporal granularity of the event slices, or \textit{TimeSliceDuration} in the following, sensor size, and filter size.

In the following we assume a fixed sensor and filter size - this includes only one type of filter (in the sense of speed selectivity).
In this specific evaluation, a sensor resolution of $128\times128$ and a filter size of $21\times21$ was used.
Furthermore, the filters were initialized with a speed selectivity of $f_0=0.08$.

Under this assumption, the runtime performance is primarily governed by to parameters of the implementation:
\begin{enumerate}
	\item Duration of an event slice
	\item No. of filter orientations
\end{enumerate}

As default values, a duration of $10\mathrm{ms}$ and 4 filter orientations are assumed.

The evaluation is presented once for the sequential, as well as for the parallel implementation.
It was performed on a computer running ArchLinux x64 on an Intel Core2Duo Processor E6400 2M Cache, 2.13 GHz, 1066 MHz FSB).
The Nvidia GT 440 graphic card with driver version 352.21 and CUDA 7.0.28-2 was used for the parallelized version.

\subsubsection{Dataset}
Since the evaluation only focuses on the time it takes to process a stream of events, there are no requirements as to available ground truth.
For this reason, the combined Dynamic Vision / RGB-D dataset from \cite{ebslamdataset} was used.
The dataset consists of 5 scene set-ups and a total of 26 takes, with lengths varying between 20 to 60 seconds and uneven motion speed (and thus event generation).

The dataset is summarized by the following table:\\
\begin{center}
\begin{tabular}{ | c | c | c | c | }
	\hline		
	Scenario & Take & Duration & Events\\
	\hline	
	\hline	
	1 & 1 & 30.9s & 1456422\\
	2 & 1 & 25.0s & 984609\\
	2 & 2 & 24.8s & 1228999\\
	2 & 3 & 25.3s & 1420190\\
	2 & 4 & 25.2s & 1303302\\
	3 & 1 & 25.1s & 1811233\\
	3 & 2 & 25.1s & 1822262\\
	3 & 3 & 36.8s & 2750400\\
	3 & 4 & 35.3s & 2739406\\
	3 & 5 & 36.8s & 2475213\\
	3 & 6 & 46.9s & 3260258\\
	3 & 7 & 50.3s & 2805818\\
	3 & 8 & 46.9s & 2046209\\
	7 & 1 & 23.5s & 1341056\\
	7 & 2 & 21.5s & 1340804\\
	7 & 3 & 30.6s & 1769307\\
	7 & 4 & 31.9s & 2203772\\
	7 & 5 & 61.7s & 1645344\\
	7 & 6 & 61.9s & 1469080\\
	7 & 7 & 42.0s & 1077844\\
	7 & 8 & 61.9s & 1680188\\
	8 & 1 & 27.5s & 1603852\\
	8 & 2 & 27.3s & 1545613\\
	8 & 3 & 30.5s & 1081577\\
	8 & 4 & 28.4s & 1143159\\                     
	\hline			
\end{tabular}
\end{center}


\subsubsection{Evaluation of Sequential Implementation}
\subsubsection{Evaluation of Parallel Implementation}
%\paragraph{Duration of event slices}
%\paragraph{No. of filter orientations}

\subsection{Qualitative:} How does the output compare to ground truth??
CHANGE THIS, but keep it in some way I guess...In this specific evaluation, a sensor resolution of $128\times128$ and a filter size of $21\times21$ was used.
Furthermore, the filters were initialized with a speed selectivity of $f_0=0.08$.
\paragraph{Ground Truth}
How did we get the ground truth, how was it produced. Highlight pit falls
\paragraph{Evaluation Metrics}
How did we measure our results
