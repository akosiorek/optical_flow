\documentclass[a4paper,twoside, openright,12pt]{report}
\usepackage{psfrag,amsbsy,graphics,float}
\usepackage{graphicx, color} %Deleted [dvips] in front of {graphicx, color} for usage also with PDFLaTex
\usepackage[latin1]{inputenc}
\usepackage{verbatim}
\usepackage{amsthm}

% for pseudocode
\usepackage{algorithm}
\usepackage{algpseudocode}
\usepackage{pifont}

\newtheorem{defwrp}{Definition}
% based on the LSR Student Template, last change: 2014-06-05

%_______Kopf- und Fußzeile_______________________________________________________
\usepackage{fancyhdr}
\pagestyle{fancy}
%um Kopf- und Fußzeile bei chapter-Seiten zu reaktivieren
\newcommand{\helv}{%
   \fontfamily{phv}\fontseries{a}\fontsize{9}{11}\selectfont}
\fancypagestyle{plain}{
	\fancyfoot{}% keine Fußzeile
	\fancyhead[RE]{\helv\leftmark}% Rechts auf geraden Seiten=innen; in \leftmark stehen \chapters
	\fancyhead[LO]{\helv\rightmark}% Links auf ungeraden Seiten=außen;in \rightmark stehen \sections
	\fancyhead[RO,LE]{\thepage}}%Rechts auf ungeraden und links auf geraden Seiten
%Kopf- und Fußzeile für alle anderen Seiten
\fancyfoot{}
\fancyhead[RE]{\helv\leftmark}
\fancyhead[LO]{\helv\rightmark}%alt:\fancyhead[LO]{\itshape\rightmark}
\fancyhead[RO,LE]{\thepage}
%________________________________________________________________________________


%_Definieren der Ränder und Längen__________
\setlength{\textwidth}{15cm}
\setlength{\textheight}{22cm}
\setlength{\evensidemargin}{-2mm}
\setlength{\oddsidemargin}{11mm}
\setlength{\headwidth}{15cm}
\setlength{\topmargin}{10mm}
\setlength{\parindent}{0pt} % Kein Einrücken beim Absatz!!
%___________________________________________

%_Hyperref for CC Url__________
\usepackage{hyperref}
%___________________________________________

%_______Title Page__________________________________________
\begin{document}
\pagestyle{empty}
\enlargethispage{4.5cm} %Damit das Titelbild weit genug unten ist!
\begin{center}
\phantom{u}
\vspace{0.5cm}
\Huge{\sc An Efficient Event-Based Optical Flow Implementation in C/C++ and CUDA}\\
\vspace{1.5cm}
		\large{
			PROJECT REPORT\\%i.e. DIPLOMA THESIS, BACHELOR THESIS, ADVANCED SEMINAR,
			%Intermediate Report\\
			\vspace{0.4cm}
			submitted by\\
			Adam Kosiorek,
			David Adrian,
			Johannes Rausch\\
			% if this is a diploma/bachelor/master thesis include the following:
			%\vspace{0.5cm}
			%born on: DD.MM.YYYY\\
			%\vspace{0.5cm}
			%Streetname XX \\
			%Zipcode City \\
			%Tel.: xxx\,xxxxxxxx \\
			\vspace{1.5cm}
			NEUROSCIENTIFIC SYSTEM THEORY\\
			Technische Universit\"at M\"unchen\\
			\vspace{0.3cm}
			Prof. Dr J\"org Conradt\\
		}
\end{center}
\vspace{5.5cm}
\begin{tabular}{ll}
Supervisor: & Dipl.-Inf. Nicolai Waniek\\
% add the start and intermediate report dates for DA/BA/MA thesis
%Start: & xx.xx.201x  \\
%Intermediate Report: &  xx.xx.201x  \\
Final Submission: &  07.07.2015 \\
\end{tabular}
%____________________________________________________________

\newpage
% \cleardoublepage

%_______Abstract_____________________________________________
\topmargin5mm
\textheight220mm
\pagenumbering{arabic}
\phantom{u}
\begin{abstract}
  Short Summary about EBOF (the algorithm), our project plan, the achieved goals.
\end{abstract}
%____________________________________________________________


\pagestyle{fancy}

%_________Inhaltsverzeichnis__________________________
\tableofcontents
%_____________________________________________________

%_________Einleitung__________________________________
\chapter{Introduction}

Very short introduction to optical flow, and highlight its relevance.
More detailed about doing event-based things, and why its a very difficult thing.

\section{Problem Statement}

What are we gonna do for this project work? Implement a specific EVENT-BASED Optical Flow algorithm.
Talk more detailed about the algorithm and the needs (parallization, massive filtering etc.).
High-level overview!

\section{Related Work}

Here we gonna talk about the paper we gonna implement (or in the section above).
But definitely give a very short summary about other approaches and implementations (if we can find any public).

%____________________________________________________



%_____Kapitel 2_________________________________
\chapter{Main Part}

\newpage
\section{Algorithm}
We might want to do a short summary of the original algorithm to outline how it works, so the following sections make sense.

\section{Our Code Design}
Summary of needs of the algorithm when looking at actual implementations not just mathematical description.
\subsection{Overview Pipeline}
As our previous class diagram, as flow chart perhaps
\subsection{Convolution Engine}
Detailed look at the convolution engine idea??
\subsection{Other components}
Other Stuff??

\section{Implementation Details}
The project is managed by CMake (min. ver. 3.2) and written in C++11. 
Vital components are extensively tested with unit tests powered by Google C++ Unit Testing Framework. 
We use Eigen for general-purpose linear algebra, in particular Eigen::SparseMatrix and Eigen::Matrix underlie EventSlice and FlowSlice classes respectively.

\subsection{Base Implementation}

Since performance is our top priority, we aim for parallel execution and optimize code where possible. 
The pipeline has 4 blocks, each of which depends on the output of the previous one. 
They cannot process the same piece of data concurrently, but they can be streamlined, such that each component will be executed asynchronously by an independent worker, with possible parallelization inside of the component.
This approach requires buffering of data between components, which we realize with blocking unbounded buffers. 

Another concern is modularity, which not only makes the code easier to maintain, but also enables extensions and an easy exchange of building blocks. 
Therefore, we abstract interfaces of each component.
To further increase it, Fourier-convolution based FilteringEngines use three objects via dependency injection: FilterFactory, which produces filters of given angle, FourierPadder, which pads data and filters to the size required by Fourier-based linear convolution, and FourierTransformer, which carries out forward and backward Fourier transformation. The default implementation of the last one relies on FFTW for efficient computation.

Another advantage of basing the algorithm on 2D Fourier convolution is that the following Event Slices do not have to be contiguous in memory (as it would be the case with 3D Fourier convolution for efficiency reasons).
It allows us to use a circular buffer to store incoming Event Slices.
When a new Event Slice comes in, the buffer is rotated and the last ES, which isn't needed any more, is overwritten with the new one.
With this scheme, all memory needed for storing Event Slices (or their intermediate representation in frequency domain) can be allocated at startup of the program, before processing any events, and thus improving run-time performance.

\subsection{CUDA Implementation}

The pseudo code of algorithm \ref{algo:optic} suggests, that the most frequent operations are element-wise matrix-matrix multiplication and addition.
Since each EventSlice enters the circular buffer of the FilteringEngine only once, and leaves only when the final optical flow is computed, the amount of data transferred is insignificant when compared with the number of floating point operations required.
CUDA usually performs very well under such conditions.
We use a singleton Cuda class for managing CUDA device handlers.
In order not to handle memory management manually, we devised the DeviceBlob class. 
It acts as a binary data container, capable of allocating and deallocating device memory as well as initializing it and copying data between device and host memory. 
The filtering algorithm uses two main arithmetic operations: (1) $Y = \alpha X + Y$, or AXPY as defined in BLAS level 1 operations and (2) $Z = X .* Y$ element-wise matrix-matrix multiplication with the result not-aliased with any of the ingredients. We realize the former by a call to CuBLAS' AXPY function and use a specialized CUDA kernel for the latter.
%\newpage
%\section{Experimental Results}
In the following, the implementation of the algorithm has been evaluated with regard to its efficiency, as well as to its accuracy with regard to ground truth. 
For the quantitative evaluation, we used large event data sets with a sensor resolution of $128\times$.
The available ground truth data for the qualitative evaluation used a sensor resolution of $240\times180$.
Because the ground truth scenes were very short, they were not suitable to be also used in the quantitative evaluation.

\section{Quantitative Evaluation}

One of the primary goals of this project was to evaluate the real-time capabilities of this algorithm.
This section concerns exclusively the evaluation of execution speed of the algorithm with respect to real-time. As a simplification, real-time shall be defined as following;
\begin{defwrp}
	Given a stream of events with duration $t_d$, it is processed in real-time, if the total processing time $t_p$ is smaller or equal to $t_d$.
\end{defwrp}

This definition is a simplified assumption of the real-time constraint as it is oblivious to the fact that many events could occur in a very short time period.
However, since we are quantizing events to discrete ES, the effects of this simplification can be neglected.
The processing speed primarily depends on the temporal granularity of event slices, or event slice duration, sensor size, and filter size.

In the following, we assume a fixed sensor and filter size.
Additionally only one type of filter parametrization (e.g., speed selectivity) is employed.
The only the variation is the orientation of it, as described in the previous chapter.
In this specific evaluation, a sensor resolution of $128\times128$ and a filter size of $21\times21$ was used.
Furthermore, the filters were initialized with a frequency selectivity of $f_0=0.08$.

Under this assumption, the runtime performance is primarily governed by two parameters of the implementation:
\begin{enumerate}
	\item Duration of an event slice
	\item Number of filter orientations
\end{enumerate}

As default values, a duration of $10\mathrm{ms}$ and 4 filter orientations are assumed.

The evaluation is presented for the parallel implementation only.
The sequential implementation produces the same results, but real-time is not achieved except for parameter settings which have \edit{no practical relevances as no viable results are produced.}{too complicated}
The quantitive analysis was performed on a computer running ArchLinux x64 on an Intel Core2Duo Processor E6400 2M Cache, 2.13 GHz, 1066 MHz FSB).
The Nvidia GT 440 graphic card with driver version 352.21 and CUDA 7.0.28-2 was used for the parallelized version.

\subsection{Dataset}
Since the evaluation only focuses on the time it takes to process a stream of events, there are no requirements as to available ground truth.
For this reason, the combined Dynamic Vision / RGB-D dataset from \cite{weikersdorfer2014event} was used.
The dataset consists of 5 scene set-ups and a total of 26 takes, with lengths varying between 20 to 60 seconds and uneven motion speed (and thus event generation).

The dataset is summarized by the following table:\\
\begin{center}
\begin{tabular}{ | c | c | c | c | }
	\hline		
	Scenario & Take & Duration & Events\\
	\hline	
	\hline	
	1 & 1 & 30.9s & 1456422\\
	2 & 1 & 25.0s & 984609\\
	2 & 2 & 24.8s & 1228999\\
	2 & 3 & 25.3s & 1420190\\
	2 & 4 & 25.2s & 1303302\\
	3 & 1 & 25.1s & 1811233\\
	3 & 2 & 25.1s & 1822262\\
	3 & 3 & 36.8s & 2750400\\
	3 & 4 & 35.3s & 2739406\\
	3 & 5 & 36.8s & 2475213\\
	3 & 6 & 46.9s & 3260258\\
	3 & 7 & 50.3s & 2805818\\
	3 & 8 & 46.9s & 2046209\\
	7 & 1 & 23.5s & 1341056\\
	7 & 2 & 21.5s & 1340804\\
	7 & 3 & 30.6s & 1769307\\
	7 & 4 & 31.9s & 2203772\\
	7 & 5 & 61.7s & 1645344\\
	7 & 6 & 61.9s & 1469080\\
	7 & 7 & 42.0s & 1077844\\
	7 & 8 & 61.9s & 1680188\\
	8 & 1 & 27.5s & 1603852\\
	8 & 2 & 27.3s & 1545613\\
	8 & 3 & 30.5s & 1081577\\
	8 & 4 & 28.4s & 1143159\\                     
	\hline			
\end{tabular}
\end{center}


%\subsubsection{Evaluation of Sequential Implementation}
%This section only deals with a small subset of testable parameters.
%This is due to the linear scaling of parameters, as well as the the fact that its far from real-time and not recommended for production usage.

\subsection{Evaluation of Parallel Implementation}
\paragraph{Duration of Event Slices}
As described before, all events are quantized and collected into ES of the duration $t_s$.
While the optic flow field is the most accurate at a resolution of $1\mu\mathrm{s}$, the resolution of the eDVS timestamps, it increases the computational effort a lot.
Our implementation approach approximates by reducing the temporal granularity and compressing events into event slices, at the cost of accuracy.
Depending on whether it is more favorable to have approximative, but real-time results, or very accurate results without time constraints is up to the use-case.
In fig. \ref{fig:gpu_tsd} we can see the development of the run-time performance with increasing durations $t_s$.
\begin{figure}[!htb]
	\centering
	\includegraphics[scale=.9]{gpu_tsd.eps}
	\caption[Variation of the duration of the event slices during the quantizing step]{Variation of the duration of the event slices during the quantizing step. Blue bar corresponds to the threshold of real-time; above the blue bar means an event stream was processed faster than the time it represents. The cross corresponds to the mean speed across all takes. The black error bars represent the standard deviation across all the takes.}
	\label{fig:gpu_tsd}
\end{figure}
\paragraph{Number of Filter Orientations}
The algorithm in its basic version utilizes one single type of spatio-temporal Gabor filter.
This means that only a single specific parametrization of the spatial and temporal component is used.

\begin{figure}[!htb]
	\centering
	\includegraphics[scale=.9]{gpu_fo.eps}
	\caption[Variation of the number of unique orientations of the spatio-temporal gabor filter.]{Variation of the number of unique orientations of the spatio-temporal gabor filter. Each orientation corresponds to a separate filter in the filter bank. Blue bar corresponds to the threshold of real-time; above the blue bar means an event stream was processed faster than the time it represents. The cross corresponds to the mean speed across all takes. The black error bars represent the standard deviation across the all takes.}
	\label{fig:gpu_fo}
\end{figure}

As we can see in fig. \ref{fig:gpu_fo}, with up to 8 filter orientations we can still easily achieve real-time.
Due to the nearly linear scaling of the performance, one can approximate the performance loss with increasing number of filter orientations.

\section{Qualitative Evaluation} 
To ensure the correctness of the algorithm and investigate effects of different filter parameters on the accuracy, we evaluated the algorithm with multiple ground truth data sets.
The provided ground truth is created for a sensor resolution of $240\times180$.
A filter size of $21\times21$ is used and the filters are initialized with a speed selectivity of $12$ pixels/s. 


We primarily use event data of simple geometric objects moving along predefined trajectories to be able to clearly identify reasons for decreased accuracy.
We firstly evaluate the correctness of the algorithm by implementing it in Matlab, due to the provided functionality in terms of mathematical methods and visualization.
By implementing scripts to read out the flow files computed by the c++ code, we can also use the Matlab evaluation scripts for the c++ output.
\subsection{Ground Truth}
For the evaluation of the algorithm, we use ground truth data from real and synthetic scenes created in a separate project \cite{Scherer2015}.
The real data is produced with a $240\times180$ DVS sensor that has a separate black-and-white camera to record regular dense frames.
An established optical flow algorithm is used on frames to create ground truth.


Synthetic data is created separately with an open-source 3D content creation program. 
In this case, the temporal resolution of ground truth frames is significantly higher. %and noise in the scene was reduced.
This allows us to perform an accurate temporal matching between our results and the ground truth.

\subsection{Evaluation Metrics}
For the qualitative evaluation, we compare the flow generated by the event-based optical flow algorithm with the corresponding ground truth flow field. As a primary metric, we investigate the angular error between the computed results and the ground truth. 

To only take into account the flow at the locations of past events, we create a mask that contains the locations of all events of increased light intensity.
This mask consists of the same number of slices as there are quantized event slices.
For each slice, the mask contains only zero-entries except for coordinates where an event occurred. 
By only considering the flow values at the non-zero positions of the mask, 
the angular error between our results and the ground truth is then computed.


For some of the synthetic data sets, due to prior knowledge about the scene, we can also make conclusions about the accuracy without the ground truth.
This is particularly the case in scenes where an object followed a linear trajectory with constant speed.
In these cases, calculating the average angle and velocity of the optical flow can even be a more reliable measure. 

\subsection{Preprocessing}
In the raw event data several defect pixels at different locations constantly trigger events in the camera.
This leads to a deteriorated performance of the algorithm, as these events cause high filter response values with random angles.
%Add pictures of plots with defect pixels and possibly compare the error

\begin{figure}[tb]
\centering
\begin{subfigure}{.45\textwidth}
  \centering
  \includegraphics[height=.5\linewidth]{figs/defect-pixels/defect-quadrat.png}
  \caption{}
\end{subfigure}
\begin{subfigure}{.45\textwidth}
  \centering
  \includegraphics[height=.5\linewidth]{figs/defect-pixels/cleaned-quadrat.png}  \caption{}
\end{subfigure}
\caption[Preprocessing step: Removing events from defect pixels.]{Comparison of the flow before and after the defect pixel clean up. Image (a) shows high responses in the left and lower left corner, which are caused by defect pixels. Through manually feeding the coordinates into an event converter script, the defect events are removed from the data and artifacts are removed from the resulting flow field (b).}
\label{fig:defect-pixel-cleanup}
\end{figure}


To obtain more meaningful results, we manually eliminate these pixels from the event data.
To do so, the event data is firstly inspected in the jAER Viewer. 
Within the software, all recorded pixel coordinates can be analyzed individually.
With the \textit{Hot Pixels} filter, locations of pixels that are triggered very often within a certain time frame can be easily determined.
%Check whether florian mentions data formats
A readout script for \textit{.aedat} files acquires the location, time and parity for all events.
By providing the script with the locations of defect pixels, erroneous events are automatically removed.
The event data is then extended with a unique ID for each event and saved as a tabular file.
%This way, the event data is readable by the c++ program, which is used to read, write and convert event data.
% and decrements the index value for all following events.

%The event data is saved in a binary format.
%To further process the data, 

\subsection{Results}
A thorough evaluation of a simple geometric scenes was conducted.
This was done to ensure the correctness of the algorithm and test its performance for different parameter choices.

In this experiment, we only inspect the locations of events of increasing intensity.
We do this, because we want to investigate the angular error in comparison to the ground truth in a meaningful way. 
At the \textit{off}-event locations, the event-based flow is pointing in opposite direction, which leads to a large offset of the average angular error. 
A close-up view that shows the described masking of a flow field is shown in Figure \ref{fig:qaudrat-close-masking}.


The ground truth provides a time stamp for each frame of the ground truth flow.
While the event data is quantized to Event Slices, a second set of time stamps is generated for each slice of events.
As the ground truth frames are not created for the exact same times as the quantized slices, we first have to determine the best fitting ground truth flow field for each slice.
For the evaluation, two different approaches of comparing the computed flow with the ground truth are applied.
Firstly, we directly compare each slice with the ground truth frame for which the difference of the two respective time stamps is smallest.
Secondly, we linearly interpolate from the two closest ground truth frames that have been recorded before and after the time of the event slice.

\subsubsection{Evaluation of Davis DVS Data}
For the evaluation of the data recorded by the Davis DVS sensor, we first remove defect pixels from the event data.

\paragraph{First DVS Scene: Moving Square}
The first scene consists of a horizontally moving rectaungular object (see Figure \ref{fig:quadrat-snapshots}).
We observe that the flow directions at the corners of the vertical edge tend to point outwards instead of pointing in $x$-direction. 
This is likely caused by the aperture problem and could be tackled in future experiments by performing a normalization step in the preprocessing.
Table \ref{tab:error_comparison_square} shows a comparison of calculated errors for the moving square. The RMSE, mean, and median angular error between the computed flow and the ground truth are calculated. 
The median error dropped to below $5^\circ$ for some parameter settings without additional normalization. 


\begin{figure}[tb]
\centering
\begin{subfigure}{.45\textwidth}
  \centering
  \includegraphics[height=.6\linewidth]{figs/quadrat_close.jpg}
  \caption{}
  \label{fig:qaudrat-close-masking-1}
\end{subfigure}
\begin{subfigure}{.45\textwidth}
  \centering
  \includegraphics[height=.6\linewidth]{figs/quadrat_close_mask.jpg}
  \caption{}
  \label{fig:qaudrat-close-masking-2}
\end{subfigure}
\begin{subfigure}{.45\textwidth}
  \centering
  \includegraphics[height=.6\linewidth]{figs/quadrat_close_GT.jpg}
  \caption{}
  \label{fig:qaudrat-close-masking-3}
\end{subfigure}
\begin{subfigure}{.45\textwidth}
  \centering
  \includegraphics[height=.6\linewidth]{figs/quadrat_close_GT_masked.jpg}
  \caption{}
  \label{fig:qaudrat-close-masking-4}
\end{subfigure}
\caption[First scene: Square moving to the right.]{A moving square was recorded with the DVS sensor.
The flow field computed by the event-based optical flow algorithm is shown in (a).
The corresponding ground truth frame is shown in Figure (c).
To evaluate the optical flow, a mask with the event locations for the corresponding slice is applied to the flow fields.
Figures (b) and (d) show the flow at these locations for the computed flow and the ground truth.
Only the flow values at the event locations are considered in the evaluation.
The flow in the ground truth is consistently directed in $x$-direction, whereas the computed flow is pointing outwards at the outer edges. This aperture effect is likely compensable through normalization.}
\label{fig:qaudrat-close-masking}
\end{figure}

\begin{figure}[tb]
\centering
\begin{subfigure}{.45\textwidth}
  \centering
  \includegraphics[height=.6\linewidth]{figs/quadrat/quadrat-1.png}
  \caption{}
\end{subfigure}
\begin{subfigure}{.45\textwidth}
  \includegraphics[height=.6\linewidth]{figs/quadrat/quadrat-masked-1.png}
  \caption{}
\end{subfigure}
\begin{subfigure}{.45\textwidth}
  \centering
  \includegraphics[height=.6\linewidth]{figs/quadrat/quadrat-GT-1.png}
  \caption{}
\end{subfigure}
\begin{subfigure}{.45\textwidth}
  \centering
  \includegraphics[height=.6\linewidth]{figs/quadrat/quadrat-GT-masked-1.png}
  \caption{}
\end{subfigure}
\caption[First scene: Robot approaching the DVS sensor.]{Second scene: Robot approaching the DVS sensor.
Figure (a) shows the computed optical flow for the scene. 
The masked flow field is shown in Figure (b).
The corresponding ground truth before and after masking is shown in Figures (c) and (d).}
\label{fig:quadrat-snapshots}
\end{figure}


\begin{table}[tb]
	\centering
		\begin{tabular}{lccccccc}
Scene & Setting & Matching & RMSE & MeanErr & MedianErr & Avg. Angle \\
\hline  \hline
quadrat & R$1$ & direct & $19.58$ & $11.66$ & $6.19$ & $1.29$ & \\
quadrat & R$1$ & interp & $20.93$ & $12.61$ & $6.53$ &  & \\
quadrat & R$2$ & direct & $19.21$ & $11.36$ & $6.19$ & $1.31$ & \\
quadrat & R$2$ & interp & $20.69$ & $12.45$ & $6.75$ &  & \\
		\end{tabular}
	\caption[First scene: Comparison of angular errors for different parameters.]{Comparison of angular errors for the moving square scene.
	The Setting attribute describes the composition of the temporal filter length as well as temporal resolution.
	 These settings can be compared in table \ref{tab:parameter_settings}. Considering the provided ground truth, parameter combination 3 lead to the lowest median error. When considering the underlying scene with vertical edges moving in $x$-direction, the average angle in setting 5 points to even better results.}
	\label{tab:error_comparison_square}
\end{table}

\begin{table}[tb]
	\centering
		\begin{tabular}{lccc}
Setting & Filter Length [ms] & Temp. Res. [ms] & Angle incr. ($0$ to $360^\circ$) \\
\hline  \hline
R$1$ & $0.70$ & $0.010$ & $45.00$\\
R$2$ & $0.70$ & $0.010$ & $30.00$\\
S$1$ & $0.20$ & $0.005$ & $45.00$\\
S$2$ & $0.20$ & $0.005$ & $30.00$\\
S$3$ & $0.03$ & $0.001$ & $45.00$\\
S$4$ & $0.03$ & $0.001$ & $30.00$\\
		\end{tabular}
	\caption[Different parameter settings for the evaluations]{Different parameter settings have been used for the evaluation of the flow.
	 The spatial size of the filter remained constant during the experiments.
	 The filter length was adjusted as well as the temporal resolution, i.e. the distance between time slices. 
	 Furthermore, the composition of the filter bank was varied by changing the rotation of the individual Gabor filters. 
	 For this, Gabor filters were rotated by angles between $0$ and $360$ degrees with constant angular increments.}
	\label{tab:parameter_settings}
\end{table}

\paragraph{Second DVS Scene: Approaching small robot}

In the second scene, a small robot is approaching the DVS sensor.
In the ground truth we can observe that the average angle of the computed flow is about $-135^\circ$ (see Figure \ref{fig:pushbot-snapshots}). 
As seen in Table \ref{tab:error_comparison_pushbot}, angular error between the event-based optical flow and the ground truth is rather high in this case.
One reason for this is the movement direction of the robot.
In contrast to the first scene, the robot does not move in any direction perpendicular to its edges.
Without additional normalization the computed optical flow points only in the directions perpendicular to the edges, which is why the actual movement direction is not reflected properly by the computed flow.
Furthermore, noise and other contours on the surface of the robot cause flow in random directions to be detected.
This further increases the calculated angular error.
The average angle of the computed flow amounts to about $-100^\circ$, which is close to average angle of the ground truth flow, considering the missing normalization.


\begin{figure}[tb]
\centering
\begin{subfigure}{.45\textwidth}
  \centering
  \includegraphics[height=.6\linewidth]{figs/pushbot/pushbot-1.png}
  \caption{}
\end{subfigure}
\begin{subfigure}{.45\textwidth}
  \includegraphics[height=.6\linewidth]{figs/pushbot/pushbot-masked-1.png}
  \caption{}
\end{subfigure}
\begin{subfigure}{.45\textwidth}
  \centering
  \includegraphics[height=.6\linewidth]{figs/pushbot/pushbot-GT-1.png}
  \caption{}
\end{subfigure}
\begin{subfigure}{.45\textwidth}
  \centering
  \includegraphics[height=.6\linewidth]{figs/pushbot/pushbot-GT-masked-1.png}
  \caption{}
\end{subfigure}
\caption[Second scene: Robot approaching the DVS sensor.]{Second scene: A small robot is approaching the DVS sensor. 
Figure (a) shows the computed optical flow for the scene. 
The masked flow field is shown in Figure (b).
The corresponding ground truth before and after masking is shown in Figures (c) and (d). 
In the computed flow, a horizontal edge at the bottom and a vertical edge at the left of the robot are clearly visible (b).
The flow pointing from these edges indicates the direction of movement. 
It is most likely not pointing to the bottom left because of the missing normalization.}
\label{fig:pushbot-snapshots}
\end{figure}

\begin{table}[tb]
	\centering
		\begin{tabular}{lccccccc}
Scene & Setting & Matching & RMSE & MeanErr & MedianErr & Avg. Angle \\
\hline  \hline
pushbot & R$1$ & direct & $94.05$ & $78.66$ & $71.46$ & -$101.32$ & \\
pushbot & R$1$ & interp & $94.34$ & $78.96$ & $71.96$ &  & \\
pushbot & R$2$ & direct & $94.12$ & $78.69$ & $71.32$ & -$101.33$ & \\
pushbot & R$2$ & interp & $94.35$ & $78.96$ & $71.98$ &  & \\
		\end{tabular}
	\caption[Second scene: Comparison of angular errors for different parameters.]{Second scene: Comparison of angular errors for different parameters. The algorithm could not properly compute the flow for movement that is not perpendicular to the object's edges. This leads to a rather high angular error for all settings.}
	\label{tab:error_comparison_pushbot}
\end{table}

\paragraph{Third DVS Scene: Movement of Sensor inside of Room}

In the third investigated scene, the sensor itself is moved while observing the interior of a room (see Figure \ref{fig:skateboard-snapshots}). 
Due to the differences in how the flow was computed for the ground truth and the event-based optical flow, the angular errors are rather high as well in this scene.

The ground truth algorithm that works on dense images computes optical flow with rather homogeneous intensity for the whole scene.
In Figure \ref{fig:skateboard-snapshots1} one can see that the edges of the walls are causing high responses and are clearly differentiable from the rest of the scene with the event-based optical flow.
Furthermore, it is notable that the algorithm was able to properly distinguish the rather high noise caused by the floor and actual edges in the scene.
Table \ref{tab:error_comparison_skateboard} shows that the angular errors for this particular scene are rather high.
The average angle of the computed flow varies significantly for different settings (see Table \ref{tab:app_error_comparison_skateboard}). 
One of the reasons for this is the rather high amount of noise, which partly leads to computed flow of arbitrary angles.
However, the figure of the masked ground truth and computed flow shows a good visual overlap for the dominant parts of the scenes.
Weighting the angular errors with the response intensity is likely to further reduce the error.


\begin{figure}[tb]
\centering
\begin{subfigure}{.45\textwidth}
  \centering
  \includegraphics[height=.6\linewidth]{figs/skateboard/skateboard-1.png}
  \caption{}
  \label{fig:skateboard-snapshots1}
\end{subfigure}
\begin{subfigure}{.45\textwidth}
  \includegraphics[height=.6\linewidth]{figs/skateboard/skateboard-masked-1.png}
  \caption{}
\end{subfigure}
\begin{subfigure}{.45\textwidth}
  \centering
  \includegraphics[height=.6\linewidth]{figs/skateboard/skateboard-GT-1.png}
  \caption{}
\end{subfigure}
\begin{subfigure}{.45\textwidth}
  \centering
  \includegraphics[height=.6\linewidth]{figs/skateboard/skateboard-GT-masked-1.png}
  \caption{}
\end{subfigure}
\caption[Third scene: Moving DVS sensor observing interior of a room.]{Third scene: Moving DVS sensor observing interior of a room. 
Figure (a) shows the computed optical flow for the scene. 
The masked flow field is shown in Figure (b).
The corresponding ground truth before and after masking is shown in Figures (c) and (d). 
Trough masking, the visual similarities between the result (b) and the ground truth (d) at the event locations become visible.
}
\label{fig:skateboard-snapshots}
\end{figure}

\begin{table}[tb]
	\centering
		\begin{tabular}{lccccccc}
Scene & Setting & Matching & RMSE & MeanErr & MedianErr & Avg. Angle \\
\hline  \hline
skateboard & R$1$ & direct & $91.93$ & $73.85$ & $61.86$ & $58.94$ & \\
skateboard & R$1$ & interp & $91.75$ & $73.47$ & $60.94$ &  & \\
skateboard & R$2$ & direct & $91.78$ & $73.75$ & $61.77$ & $58.37$ & \\
skateboard & R$2$ & interp & $91.58$ & $73.36$ & $61.02$ &  & \\
		\end{tabular}
	\caption[Third scene: Comparison of angular errors for different parameters.]{Third scene: Comparison of angular errors for different parameters. 
	Due to a high degree of noise, the angular error is rather high and the average angle greatly varies.}
	\label{tab:error_comparison_skateboard}
\end{table}

\subsubsection{Synthetic Data}

The following evaluations have been conducted with a synthetically created set of event data and ground truths.
Due to the method of creating the scenes with a 3D software, the scenes are rather short ($0.3$ s) and the depicted geometrical objects move faster.
The temporal resolution for the filtering is increased in order to deal with the higher number of events due to the fast movement.
The responses of the event-based flow are very low in the three investigated cases. 
This is likely caused by the very high movement speed of the observed object.
The Gabor filters we used in the experiments have a fixed speed selectivity. 
If the movement speed of objects in the scene differs a lot from this speed, the filter response is rather low.
To still properly visually compare the computed flow fields, we magnify the values in the quiver plot by $10$ for the 'square1.2' and 'square2' scene, and by $100$ for the 'baelle' scene.

\paragraph{First Synthetic Scene: Square moving linearly}


In the first synthetic scene, a square is linearly moving from the middle of the window to the lower right corner of the window with constant speed.
It is interesting to note that the algorithm fails to recognize the right and bottom edge of the square (see Figure \ref{fig:square12-snapshots}).
This is likely caused by the perfectly aligned simultaneous movement off all pixels on the outer edges.
%Explain further
The object is moving in a direction not perpendicular to its edges.
As in the previous scenes, the angular error is likely increased due to the missing normalization and amounts to about $40^\circ$ for different measurement metrics (see Table \ref{tab:error_comparison_square12}). 
 
Furthermore, the effects of the aperture are visible again at the outer corners of the edges.

\begin{figure}[tb]
\centering
\begin{subfigure}{.45\textwidth}
  \centering
  \includegraphics[height=.6\linewidth]{figs/square12/square12-1.png}
  \caption{}
\end{subfigure}
\begin{subfigure}{.45\textwidth}
  \includegraphics[height=.6\linewidth]{figs/square12/square12-masked-1.png}
  \caption{}
\end{subfigure}
\begin{subfigure}{.45\textwidth}
  \centering
  \includegraphics[height=.6\linewidth]{figs/square12/square12-GT-1.png}
  \caption{}
\end{subfigure}
\begin{subfigure}{.45\textwidth}
  \centering
  \includegraphics[height=.6\linewidth]{figs/square12/square12-GT-masked-1.png}
  \caption{}
\end{subfigure}
\caption[Fourth scene: Synthetic data of a linearly moving square.]{Fourth scene: Synthetic data of a linearly moving square.
Figure (a) shows the computed optical flow for the scene. 
The masked flow field is shown in Figure (b).
The corresponding ground truth before and after masking is shown in Figures (c) and (d). 
Due to the synthetic nature of the data, all pixels of the outer edges moved simultaneously.
This leads to our algorithm failing to recognize the edges at all.
The effects of the aperture problem are visible a the right and bottom corner of the edges in Figure (a).
}
\label{fig:square12-snapshots}
\end{figure}

\begin{table}[tb]
	\centering
		\begin{tabular}{lccccccc}
Scene & Setting & Matching & RMSE & MeanErr & MedianErr & Avg. Angle \\
\hline  \hline
square1.2 & S$1$ & direct & $42.71$ & $40.86$ & $39.17$ & -$37.72$ & \\
square1.2 & S$1$ & interp & $42.84$ & $40.89$ & $39.17$ &  & \\
square1.2 & S$2$ & direct & $42.55$ & $40.78$ & $39.10$ & -$38.04$ & \\
square1.2 & S$2$ & interp & $42.65$ & $40.81$ & $39.11$ &  & \\
		\end{tabular}
	\caption[Fourth scene: Comparison of angular errors for different parameters.]{Fourth scene: Comparison of angular errors for different parameters.
	The angular error is rather high for all measurement metrics.
	A likely reason is the movement direction of the square, which is non-perpendicular to its edges.}
	\label{tab:error_comparison_square12}
\end{table}

\paragraph{Second Synthetic Scene: Square following in circular trajectory}

In the second synthetic scene, a square is moving in a circular trajectory (see Figure \ref{fig:square2-snapshots}).
The square is moving such that its front edge is always perpendicular to the direction of movement.
In comparison to the fourth scene, the angular error is greatly reduced, which emphasizes the effect of the normalization (see Table \ref{tab:error_comparison_square2}).
Again, the algorithm fails to recognize the outer edge in movement direction.

\begin{figure}[tb]
\centering
\begin{subfigure}{.45\textwidth}
  \centering
  \includegraphics[height=.6\linewidth]{figs/square2/square2-1.png}
  \caption{}
\end{subfigure}
\begin{subfigure}{.45\textwidth}
  \includegraphics[height=.6\linewidth]{figs/square2/square2-masked-1.png}
  \caption{}
\end{subfigure}
\begin{subfigure}{.45\textwidth}
  \centering
  \includegraphics[height=.6\linewidth]{figs/square2/square2-GT-1.png}
  \caption{}
\end{subfigure}
\begin{subfigure}{.45\textwidth}
  \centering
  \includegraphics[height=.6\linewidth]{figs/square2/square2-GT-masked-1.png}
  \caption{}
\end{subfigure}
\caption[Fifth scene: Synthetic data of a circularly moving square.]{Fifth scene: Synthetic data of a circularly moving square.
Figure (a) shows the computed optical flow for the scene. 
The masked flow field is shown in Figure (b).
The corresponding ground truth before and after masking is shown in Figures (c) and (d).}
\label{fig:square2-snapshots}
\end{figure}


\begin{table}[tb]
	\centering
		\begin{tabular}{lccccccc}
Scene & Setting & Matching & RMSE & MeanErr & MedianErr & Avg. Angle \\
\hline  \hline
square2 & S$1$ & direct & $32.57$ & $25.10$ & $19.81$ & -$150.44$ & \\
square2 & S$1$ & interp & $32.92$ & $25.25$ & $19.79$ &  & \\
square2 & S$2$ & direct & $32.77$ & $25.00$ & $19.31$ & -$150.63$ & \\
square2 & S$2$ & interp & $33.19$ & $25.18$ & $19.30$ &  & \\
		\end{tabular}
	\caption[Fifth scene: Comparison of angular errors for different parameters.]{Fifth scene: Comparison of angular errors for different parameters.The angular error is rather low, considering the speed and non-linearity of the object's movement.
	A reason for this is the greatly reduced influenced effect of the missing normalization step.}
	\label{tab:error_comparison_square2}
\end{table}

\paragraph{Third Synthetic Scene: Two moving balls}

The third synthetic scene depicts two balls moving around non-linear trajectories (see Figure \ref{fig:baelle-snapshots}). 
In Figure \ref{fig:baelle-snapshots1} it can be observed that the computed flow is influenced by the shading effect that was applied in the 3D scene. 

Due to the shading, the algorithm detects an additional edge inside one of the balls, which leads to more ambiguous results.
However, Table \ref{tab:error_comparison_baelle} shows that the median angular error is still rather low in comparison to other scenes.

A reason for this might be the shape of the balls. 
Without normalization, a part of the surface always roughly points in the movement direction, which leads to valid results for the corresponding event locations.

\begin{figure}[tb]
\centering
\begin{subfigure}{.45\textwidth}
  \centering
  \includegraphics[height=.6\linewidth]{figs/baelle/baelle-1.png}
  \caption{}
\label{fig:baelle-snapshots1}
\end{subfigure}
\begin{subfigure}{.45\textwidth}
  \includegraphics[height=.6\linewidth]{figs/baelle/baelle-masked-1.png}
  \caption{}
\end{subfigure}
\begin{subfigure}{.45\textwidth}
  \centering
  \includegraphics[height=.6\linewidth]{figs/baelle/baelle-GT-1.png}
  \caption{}
\end{subfigure}
\begin{subfigure}{.45\textwidth}
  \centering
  \includegraphics[height=.6\linewidth]{figs/baelle/baelle-GT-masked-1.png}
  \caption{}
\end{subfigure}
\caption[Sixth scene: Synthetic data of two balls moving through the scene.]{Sixth scene: Synthetic data of two balls moving through the scene.
Figure (a) shows the computed optical flow for the scene. 
The masked flow field is shown in Figure (b).
The corresponding ground truth before and after masking is shown in Figures (c) and (d). 
Due to a shading effect, two edges are detected for the upper sphere by the algorithm (a).}
\label{fig:baelle-snapshots}
\end{figure}

\begin{table}[tb]
	\centering
		\begin{tabular}{lccccccc}
Scene & Setting & Matching & RMSE & MeanErr & MedianErr & Avg. Angle \\
\hline  \hline
baelle & S$3$ & direct & $43.46$ & $33.57$ & $26.39$ & $13.11$ & \\
baelle & S$3$ & interp & $43.44$ & $33.48$ & $26.10$ &  & \\
baelle & S$4$ & direct & $42.88$ & $33.38$ & $26.61$ & $13.41$ & \\
baelle & S$4$ & interp & $42.81$ & $33.26$ & $26.37$ &  & \\
		\end{tabular}
	\caption[Sixth scene: Comparison of angular errors for different parameters.]{Sixth scene: Comparison of angular errors for different parameters.
	The median angular error is rather low compared to other investigated scenes. 
	This could be caused by the reduced effect of missing normalization due to the circular shape of the objects.}
	\label{tab:error_comparison_baelle}
\end{table}

\subsubsection{Wrap-Up}
Overall, the results of the evaluation seem promising.
A major part of the angular errors can be accounted for by the missing normalization in the post-processing.
Furthermore, the filter bank only incorporated Gabor filters with a fixed speed selectivity.
The problem of simultaneous movement of all pixels of an edge did only occur in the synthetic data and is unlikely to happen in real scenarios.
Nevertheless it should be addressed in further investigations.

%\section{Discussion}
%Dicuss results and look at whether they support the original paper and our code (ye, cause its so fck great :)


%_______________________________________________


%_____Zusammenfassung, Ausblick_________________________________
\chapter{Conclusion}

Don't leave it at the discussion: discuss what you/the reader can learn from the results. Draw some real conclusions. Separate discussion/interpretation of the results clearly from the conclusions you draw from them. (So-called "conclusion creep" tends to upset reviewers. It means surrendering your scientific objectivity.) Identify all shortcomings/limitations of your work, and discuss how they could be fixed ("future work"). It is not a sign of weakness of your work, if you clearly analyse and state the limitations. Informed readers will notice them anyway and draw their own conclusions, if not addressed properly.
\cite[p.~1]{Elphinstone2014}

%_______________________________________________________________


%_____Abbildungsverzeichnis_________________________________
\cleardoublepage
\addcontentsline{toc}{chapter}{List of Figures}
\listoffigures 	 %Abbildungsverzeichnis

%___________________________________________________________

%_____Literaturverzeichnis_________________________________
\cleardoublepage
\addcontentsline{toc}{chapter}{Bibliography}
\bibliography{ebof.bib}
\bibliographystyle{alphaurl}
%__________________________________________________________


%_____License_________________________________
\cleardoublepage
\chapter*{License}
\markright{LICENSE}
The MIT License (MIT)\\

Copyright (c) 2015 Adam Kosiorek, David Adrian, Johannes Rausch\\

Permission is hereby granted, free of charge, to any person obtaining a copy of this software and associated documentation files (the "Software"), to deal in the Software without restriction, including without limitation the rights to use, copy, modify, merge, publish, distribute, sublicense, and/or sell copies of the Software, and to permit persons to whom the Software is furnished to do so, subject to the following conditions:\\

The above copyright notice and this permission notice shall be included in all copies or substantial portions of the Software.\\

THE SOFTWARE IS PROVIDED "AS IS", WITHOUT WARRANTY OF ANY KIND, EXPRESS OR IMPLIED, INCLUDING BUT NOT LIMITED TO THE WARRANTIES OF MERCHANTABILITY, FITNESS FOR A PARTICULAR PURPOSE AND NONINFRINGEMENT. IN NO EVENT SHALL THE AUTHORS OR COPYRIGHT HOLDERS BE LIABLE FOR ANY CLAIM, DAMAGES OR OTHER LIABILITY, WHETHER IN AN ACTION OF CONTRACT, TORT OR OTHERWISE, ARISING FROM, OUT OF OR IN CONNECTION WITH THE SOFTWARE OR THE USE OR OTHER DEALINGS IN THE SOFTWARE.
%__________________________________________________________

\end{document}
