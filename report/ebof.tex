\documentclass[a4paper,twoside, openright,12pt]{report}
\usepackage{psfrag,amsbsy,graphics,float}
\usepackage{graphicx, color} %Deleted [dvips] in front of {graphicx, color} for usage also with PDFLaTex
\usepackage[latin1]{inputenc}
\usepackage{verbatim}

% based on the LSR Student Template, last change: 2014-06-05

%_______Kopf- und Fußzeile_______________________________________________________
\usepackage{fancyhdr}
\pagestyle{fancy}
%um Kopf- und Fußzeile bei chapter-Seiten zu reaktivieren
\newcommand{\helv}{%
   \fontfamily{phv}\fontseries{a}\fontsize{9}{11}\selectfont}
\fancypagestyle{plain}{
	\fancyfoot{}% keine Fußzeile
	\fancyhead[RE]{\helv\leftmark}% Rechts auf geraden Seiten=innen; in \leftmark stehen \chapters
	\fancyhead[LO]{\helv\rightmark}% Links auf ungeraden Seiten=außen;in \rightmark stehen \sections
	\fancyhead[RO,LE]{\thepage}}%Rechts auf ungeraden und links auf geraden Seiten
%Kopf- und Fußzeile für alle anderen Seiten
\fancyfoot{}
\fancyhead[RE]{\helv\leftmark}
\fancyhead[LO]{\helv\rightmark}%alt:\fancyhead[LO]{\itshape\rightmark}
\fancyhead[RO,LE]{\thepage}
%________________________________________________________________________________


%_Definieren der Ränder und Längen__________
\setlength{\textwidth}{15cm}
\setlength{\textheight}{22cm}
\setlength{\evensidemargin}{-2mm}
\setlength{\oddsidemargin}{11mm}
\setlength{\headwidth}{15cm}
\setlength{\topmargin}{10mm}
\setlength{\parindent}{0pt} % Kein Einrücken beim Absatz!!
%___________________________________________

%_Hyperref for CC Url__________
\usepackage{hyperref}
%___________________________________________

%_______Title Page__________________________________________
\begin{document}
\pagestyle{empty}
\enlargethispage{4.5cm} %Damit das Titelbild weit genug unten ist!
\begin{center}
\phantom{u}
\vspace{0.5cm}
\Huge{\sc An Efficient Event-Based Optical Flow Implementation in C/C++ and CUDA}\\
\vspace{1.5cm}
		\large{
			PROJECT REPORT\\%i.e. DIPLOMA THESIS, BACHELOR THESIS, ADVANCED SEMINAR,
			%Intermediate Report\\
			\vspace{0.4cm}
			submitted by\\
			Adam Kosiorek,
			David Adrian,
			Johannes Rausch\\
			% if this is a diploma/bachelor/master thesis include the following:
			%\vspace{0.5cm}
			%born on: DD.MM.YYYY\\
			%\vspace{0.5cm}
			%Streetname XX \\
			%Zipcode City \\
			%Tel.: xxx\,xxxxxxxx \\
			\vspace{1.5cm}
			NEUROSCIENTIFIC SYSTEM THEORY\\
			Technische Universit\"at M\"unchen\\
			\vspace{0.3cm}
			Prof. Dr J\"org Conradt\\
		}
\end{center}
\vspace{5.5cm}
\begin{tabular}{ll}
Supervisor: & Dipl.-Inf. Nicolai Waniek\\
% add the start and intermediate report dates for DA/BA/MA thesis
%Start: & xx.xx.201x  \\
%Intermediate Report: &  xx.xx.201x  \\
Final Submission: &  07.07.2015 \\
\end{tabular}
%____________________________________________________________

\newpage
% \cleardoublepage

%_______Abstract_____________________________________________
\topmargin5mm
\textheight220mm
\pagenumbering{arabic}
\phantom{u}
\begin{abstract}
  Short Summary about EBOF (the algorithm), our project plan, the achieved goals.
\end{abstract}
%____________________________________________________________


\pagestyle{fancy}

%_________Inhaltsverzeichnis__________________________
\tableofcontents
%_____________________________________________________

%_________Einleitung__________________________________
\chapter{Introduction}

Very short introduction to optical flow, and highlight its relevance.
More detailed about doing event-based things, and why its a very difficult thing.

\section{Problem Statement}

What are we gonna do for this project work? Implement a specific EVENT-BASED Optical Flow algorithm.
Talk more detailed about the algorithm and the needs (parallization, massive filtering etc.).
High-level overview!

\section{Related Work}

Here we gonna talk about the paper we gonna implement (or in the section above).
But definitely give a very short summary about other approaches and implementations (if we can find any public).

%____________________________________________________



%_____Kapitel 2_________________________________
\chapter{Main Part}

\section{Algorithm}
We might want to do a short summary of the original algorithm to outline how it works, so the following sections make sense.

\section{Our Code Design}
Summary of needs of the algorithm when looking at actual implementations not just mathematical description.
\subsection{Overview Pipeline}
As our previous class diagram, as flow chart perhaps
\subsection{Convolution Engine}
Detailed look at the convolution engine idea??
\subsection{Other components}
Other Stuff??

\section{Implementation}
Describe our whole C++11 impl in details
\subsection{Sequential}
\subsection{Parallel Implementation}

\section{Experimental Results}
Whole lot of evaluation.

\subsection{Ground Truth}
How did we get the ground truth, how was it produced. Highlight pit falls
\subsection{Evaluation Metrics}
How did we measure our results

\subsection{Results}
\paragraph{Qualitative:} How does it run, without regards to output quality (simply a measurement of runtimes with parameter fiddling)
Comparision CPU vs GPU etc.

\paragraph{Quantitative:} How does the output compare to ground truth??

\section{Discussion}
Dicuss results and look at whether they support the original paper and our code (ye, cause its so fck great :)

%_______________________________________________


%_____Zusammenfassung, Ausblick_________________________________
\chapter{Conclusion}

Don't leave it at the discussion: discuss what you/the reader can learn from the results. Draw some real conclusions. Separate discussion/interpretation of the results clearly from the conclusions you draw from them. (So-called "conclusion creep" tends to upset reviewers. It means surrendering your scientific objectivity.) Identify all shortcomings/limitations of your work, and discuss how they could be fixed ("future work"). It is not a sign of weakness of your work, if you clearly analyse and state the limitations. Informed readers will notice them anyway and draw their own conclusions, if not addressed properly.
\cite[p.~1]{Elphinstone2014}

%_______________________________________________________________


%_____Abbildungsverzeichnis_________________________________
\cleardoublepage
\addcontentsline{toc}{chapter}{List of Figures}
\listoffigures 	 %Abbildungsverzeichnis

%___________________________________________________________

%_____Literaturverzeichnis_________________________________
\cleardoublepage
\addcontentsline{toc}{chapter}{Bibliography}
\bibliography{ebof.bib}
\bibliographystyle{alphaurl}
%__________________________________________________________


%_____License_________________________________
\cleardoublepage
\chapter*{License}
\markright{LICENSE}
The MIT License (MIT)\\

Copyright (c) 2015 Adam Kosiorek, David Adrian, Johannes Rausch\\

Permission is hereby granted, free of charge, to any person obtaining a copy of this software and associated documentation files (the "Software"), to deal in the Software without restriction, including without limitation the rights to use, copy, modify, merge, publish, distribute, sublicense, and/or sell copies of the Software, and to permit persons to whom the Software is furnished to do so, subject to the following conditions:\\

The above copyright notice and this permission notice shall be included in all copies or substantial portions of the Software.\\

THE SOFTWARE IS PROVIDED "AS IS", WITHOUT WARRANTY OF ANY KIND, EXPRESS OR IMPLIED, INCLUDING BUT NOT LIMITED TO THE WARRANTIES OF MERCHANTABILITY, FITNESS FOR A PARTICULAR PURPOSE AND NONINFRINGEMENT. IN NO EVENT SHALL THE AUTHORS OR COPYRIGHT HOLDERS BE LIABLE FOR ANY CLAIM, DAMAGES OR OTHER LIABILITY, WHETHER IN AN ACTION OF CONTRACT, TORT OR OTHERWISE, ARISING FROM, OUT OF OR IN CONNECTION WITH THE SOFTWARE OR THE USE OR OTHER DEALINGS IN THE SOFTWARE.
%__________________________________________________________

\end{document}
